% 2015-10-01 - Matthew A. Wolff - Thesis research presentation

\documentclass{beamer}

\mode<presentation> {
	% Themes and color schemes:
	\usetheme{default}
	\usecolortheme{rose}
}

% Need this on campus machine (but not at home)
\usepackage{sansmathaccent}
\pdfmapfile{+sansmathaccent.map}

\usepackage{graphicx} % Allows including images
\usepackage{booktabs} % Allows the use of \toprule, \midrule and \bottomrule in tables

%------------------------------------------------------------------------------
%	TITLE PAGE
%------------------------------------------------------------------------------

% The short title appears at the bottom of every slide, the full title is only on the title page
\title[Multilevel Summation Method]{Multilevel Summation Method} 

\author{Matthew A. Wolff} % Your name
\institute[Purdue University] % Your institution as it will appear on the bottom of every slide, may be shorthand to save space
{
	Purdue University \\ % Your institution for the title page
	\medskip
	\textit{wolff1@purdue.edu} % Your email address
}
\date{October 1, 2015} % Could also use \today, if desired

% Add slide numbers
\setbeamerfont{page number in head/foot}{size=\small}
\setbeamertemplate{footline}[frame number]

\begin{document}

\begin{frame}
	\titlepage % Print the title page as the first slide
\end{frame}

\begin{frame}
	\frametitle{Overview} % Table of contents slide, comment this block out to remove it
	\tableofcontents % Throughout your presentation, if you choose to use \section{} and \subsection{} commands, these will automatically be printed on this slide as an overview of your presentation
\end{frame}

\section{Timeline}

\frame{
	\frametitle{Timeline}

	\tiny{
		\begin{itemize}
			\item 8/27 - Last presentation (I think)
			\item 8/31 - Softening function experimentation
			\item 9/3 - Discussed 1D interpolation errors, moving to 6D
			\item 9/7 - Read Reimer about spline interpolation error on equidistant grids
			\item 9/11 - Revamped FMM Cilk code
			\item 9/14 - Read Richards about Lebesgue constant for cardinal spline interpolation
			\item 9/14 - Wrote MATLAB code to determine $h$ for given $p$ and $\tau$
			\item 9/15 - Choosing $k$ based on $p$ for quasi-interpolation
			\item 9/18 - Decided that $a$ should be defined before $h, p, k,$ etc
			\item 9/22 - Worked on how to determine $a$
			\item 9/24 - Read about clustering, k-NN algorithms, etc
			\item 9/29 - Worked on strategy for computing "optimal" $a$
			\item 9/30 - More work on "optimal" $a$ and space-filling curves			
		\end{itemize}
	}
}

%\small
\tiny

\section{Main Ideas}

\frame{
	\frametitle{Main Ideas}

	\begin{itemize}
		\item use error analysis to choose $h$
		\item error analysis depends on $a$
		\begin{itemize}
			\item Historically, $a$ has been a multiple of $h$, i.e., $a = 5h$
		\end{itemize}
		\item We should pick $a$ first as computation of long-range doesn't really depend strongly on it
		\item Pick largest $a$ possible, such that any bin can have no more than $\sqrt{N}$ particles within it
		\begin{itemize}
			\item We want the largest $a$ because it will reduce the interpolation error estimate
			\item This should allow us to choose a larger $h$ and lower $p$ (hopefully optimizing performance)
			\item In serial, large $a$ should have less overhead when looping over bins
			\item In parallel, large $a$ should reduce number of bins and IPC
		\end{itemize}
		\item How to pick $a$?
		\item How to pick $h$?
		\item How to pick $p$?
		\item How to pick $k$?
		\item How to pick $\mu$?
	\end{itemize}
}

\section{Stuff}

\frame{
	\frametitle{Stuff}
}

\frame{
	\frametitle{The End!}
	\begin{center}
		Thank you!
	\end{center}
}

\end{document}